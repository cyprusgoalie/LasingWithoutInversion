\documentclass{article}
\usepackage[utf8]{inputenc} %useful to type directly diacritic characters
\usepackage{hyperref}
\title{Lasing Without Inversion}
\author{Berke Vow Ricketti}
\begin{document}
\maketitle{}


\bibliographystyle{ieeetr}
\bibliography{references}
\section{Abstract (100 words)}

\section{Introduction (340 words)}

In 1960, Theodore Maiman invented the world's first of light amplification by stimulated emission of radiation apparatus; today commonly known as the laser. The light created by this device was simultaneously spatially and temporally coherent, heralding in new frontiers of scientific research. 

While most modern lasers require multiple energy levels within their gain medium, a simple two level system can be used in understanding laser theory. Within the laser itself, atoms in the gain are constantly and randomly absorbing photons, raising their energy level, and spontaneously emitting photons, lowering their energy level. In order for the device to reach a threshold at which more photons are being emitted than absorbed, the number of photons in the excited stat must be higher than those in the lower, relaxed states. Since most of the atoms start in the relaxed state, this phenomenon is known as population inversion, and has been a key characteristic of most laser systems implemented throughout history.

Conventional laser systems which utilize population inversion effectively create a coherent field at a lower frequency than its incoherent energy source. When only considering single photon interactions and incoherent optical pumping, population inversion about the lasing threshold is required. However, if more complicated systems are considered, population inversion no longer becomes a requirement to achieve lasing.

The first theoretical proposal of lasing without inversion (LwI) was in 1957 by Javan, utilizing a three-level system. Since this original inception, two-level and other three-level and multi-level schemes for achieving LwI have been proposed.

Conventional, population inversion driven lasers are unable to achieve sufficient inversion at low wavelengths. This is not a problem for LwI systems, allowing the development of X-Ray and $\gamma$-ray lasers.

In Section 2 of this work, theoretical methods of achieving of LwI and Amplification without Inversion (AwI) in two-level, three-level, and multi-level systems will be explored. In Section 3, the first experiments to confirm AwI will be discussed. In Section 4, modern applications of LwI and AwI will be described, followed by a look into the state-of-the-art and future developments of the technology in Section 5, before concluding in Section 6.

\section{Theoretical Methodology (750)}
\subsection{2-level system) (250 words)}
\subsubsection{Recoil Induced lasing}

The concept of recoil induced lasing comes through a simple considering of the conservation laws of energy and momentum. Consider a two-level system where an atom may either absorb a photon and be excited to a higher energy start, or emit a photon and relax down to a lower energy state. In both instances, the atom feels is subject to a recoil and change in its momentum due to the momentum of the photon.

Consider a photon incident upon an atom in the ground state, right before it is absorbed. The frequency of the photon, $\omega_{a}$ is given by:

\begin{equation}
\omega_{a}=\omega+\Delta\omega
\end{equation}\label{eq:1}


where $\omega$ is the frequency corresponding to the atom's energy gap between its ground state and excited state and $\Delta\omega$ is the energy corresponding to the recoil, or momentum change, of the atom, defined as:

\begin{equation}
\Delta\omega = frac{h\omega^{2}}{2mc^{2}}
\end{equation}\label{eq:2}


where $h$ is planck's constant, $m$ is the mass of the atom, and $c$ is the speed of light. As such, this is almost a relativistic consideration, as we bring the atom's mass into play when considering the energy.

Let us now consider the other instance, where a photon is spontaneously emitted from an atom its excited state. Once again, the frequency of the emitted photon will depend on both the energy gap and the momentum change of the atom:

\begin{equation}
\omega_{e}=\omega-\Delta\omega
\end{equation}\label{eq:3}

Taking the difference between \hyperref[eq:2]{Eqs 2} and \hyperref[eq:3]{3} gives us the difference between the absorption and emition frequencies.

\begin{equation}
\Delta\omega = \frac{h\omega^2}{mc^2}
\end{equation}\label{eq:4}


\subsubsection{Coherently driven 2-level system}
\subsection{Amplification without Inversion (AwI) 3-level system (250 words)}
\subsection{Amplification without Inversion (AwI) Multi-level system (250 words)}

\section{Experimental Confirmation (50 words)}
\subsection{First confirmation of AwI, Zeeman coherence (250 words)}
Nottelmann et al. 1993 \cite{Nottelmann1993}
\subsection{2nd confirmation of AwI, D1 line atomic sodium (250 words)}
Fry et al. 1993 \cite{PhysRevLett.70.3235}
\subsection{3rd confirmation of AwI, Cadmium Vapour (250 words)}
van der Veer et al. 1993 \cite{PhysRevLett.70.3243}

\section{Applications (750 words)}
\subsection{Short wavelength lasers}
\subsubsection{X-ray lasers (250 words)}
\subsubsection{gamma - ray lasers (250 words)}
\subsection{Electromagnetically Induced Transparency (250 words)}

\section{State-of-the-art and Future Development (300 words)}

\section{Conclusion (100 words)}

Word count: 3050

\end{document}
