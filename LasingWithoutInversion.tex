\documentclass{article}
\usepackage[utf8]{inputenc} %useful to type directly diacritic characters
%\usepackage{hyperref}
\title{Lasing Without Inversion}
\author{Berke Vow Ricketti}
\begin{document}
\maketitle{}


\bibliographystyle{ieeetr}
\bibliography{references}
\section{Abstract ($\sim$100 words)}

\section{Introduction ($\sim$300 words)}

In 1960, Theodore Maiman invented the world's first of light amplification by stimulated emission of radiation apparatus; today commonly known as the laser. The light created by this device was simultaniously spatially and temporaly coherent, heralding in new frontiers of scientific research. 

While most modern lasers require multiple energy levels within their gain medium, a simple two level system can be used in understanding laser theory. Within the laser itself, atoms in the gain are constantly and randomly absorbing photons, raising their energy level, and spontaneously emitting photons, lowering their energy level. In order for the device to reach a threshold at which more photons are being emitted than absorbed, the number of photons in the excited stat must be higher than those in the lower, relaxed states. Since most of the atoms start in the relaxed state, this phenomenon is known as population inversion, and has been a key characteristic of most laser systems implemented throughout history.

Conventional laser systems which utilize population inversion effectively create a coherent field at a lower frequency than its incoherent energy source. When only considering single photon interactions and incoherent optical pumping, population inversion about the lasing threshold is required. However, if more complicated systems are considered, population inversion no longer becomes a requirement to achieve lasing.

\cite{Harris1989,Scully1994,Kocharovskaya1986,Mandel1993,Kilin2008,Ukhanov1999,Zhu1992,Marthaler2011,Mompart2000}
\section{Theoretical Methodology ($\sim$750)}
\subsection{Normal Laser with Inversion 2-level system) ($\sim$250 words)}
\subsection{Amplification without Inversion (AwI) 3-level system ($\sim$250 words)}
\subsection{Amplification without Inversion (AwI) Multi-level system ($\sim$250 words)}

\section{Experimental Confirmation ($\sim$750 words)}
\subsection{First confirmation of AwI, Zeeman coherence ($\sim$250 words)}
Nottelmann et al. 1993 \cite{Nottelmann1993}
\subsection{2nd confirmation of AwI, D\textsubscript{1} line atomic sodium ($\sim$250 words)}
Fry et al. 1993 \cite{PhysRevLett.70.3235}
\subsection{3rd confirmation of AwI, Cadmium Vapour ($\sim$250 words)}
van der Veer et al. 1993 \cite{PhysRevLett.70.3243}

\section{Applications ($\sim$750 words)}
\subsection{Short wavelength lasers}
\subsubsection{X-ray lasers ($\sim$250 words)}
\subsubsection{$\gamma$ - ray lasers ($\sim$250 words)}
\subsection{Electromagnetically Induced Transparency ($\sim$250 words)}

\section{State-of-the-art and Future Development ($\sim$300 words)}

\section{Conclusion ($\sim$100 words)}

Word count: $\sim$3050

\end{document}
